\documentclass{ltxdoc}
\usepackage{odsfile,tgschola,metalogo,hyperref,xspace,microtype,showexpl,booktabs,url}
\author{Michal Hoftich (\url{michal.h21@gmail.com})}
\title{The \textsf{odsfile} package:\\
inserting \textsf{opendocument spreadsheet} as \LaTeX\ tables\thanks{Version 0.1, last revisited 2012-07-22.}
}
\usepackage[english]{babel}
\lstloadlanguages{[LaTeX]Tex} 
\lstset{% 
    basicstyle=\ttfamily, 
    commentstyle=\itshape\ttfamily, 
    showspaces=false, 
    showstringspaces=false, 
    breaklines=true, 
    breakautoindent=true, 
    captionpos=t 
} 

\newcommand\ods{\textsf{ods}\xspace}

\begin{document}
\maketitle

\tableofcontents

\section{Introduction}

This is \LuaLaTeX{} package and \textsf{lua} library for reading \textsf{opendocument spreadsheet} (\ods) documents from Open/Libre Office Calc and typeseting them as \LaTeX{} tables. 
\ods format consist of number of |xml| files packed in the |zip| file. This package uses \LuaTeX's zip library and scripting to read |xml| content from this archive, 
which means that it is not possible to use this package with pdf\LaTeX{} or \XeLaTeX. On the other side, |odsfile.lua| library
can be used from Plain\TeX, Con\TeX{}t or pure |lua| scripts.   

Creation of this package was motivated by question\footnote{\url{http://tex.stackexchange.com/questions/60378/insert-libreoffice-table-as-input}} on site \url{http://tex.stackexchange.com/}. Development version of the package can be found at \url{https://github.com/michal-h21/odsfile}, all contributions and comments are welcome. 
\section{Usage}

You can load |odsfile| classicaly with
\begin{verbatim}
\usepackage{odsfile}
\end{verbatim}
There are two macros:
\begin{itemize}
\item \cmd{\includespread}
\item \cmd{\tabletemplate}
\end{itemize}

\noindent Main command is\marginpar{\cmd{\includespread}} \cmd{\includespread}. It's syntax is:\\ 
\cmd{\includespread}\oarg{key-value list}

Options are:

\begin{description}
\item[file] Filename of file to be loaded. You should specify this only on first use of \cmd{\includespread}.
\item[sheet] Name of sheet to be loaded. If it's not specified on first use of \cmd{\includespread}, then first sheet from the file is loaded. The sheet remains selected until another use of |sheet|.
\begin{LTXexample}
\begin{tabular}{l l}
\includespread[file=pokus.ods,sheet=List2]
\end{tabular}     
\end{LTXexample}
\item[range] Selects range from table to be inserted. Range is specified in format similar to spreadsheet processors, like |a2:c4|, selecting cells starting at first column, second row and ending and third column, fourth row.
\begin{LTXexample}
\begin{tabular}{lll}
\includespread[sheet=List1,range=a2:c4]
\end{tabular}     
\end{LTXexample}
You can omit some or both of the numbers:
\begin{LTXexample}
\begin{tabular}{lll}
\includespread[range=a:c4]
\end{tabular}     
\end{LTXexample}

\begin{LTXexample}
\begin{tabular}{ll}
\includespread[range=a:b]
\end{tabular}     
\end{LTXexample}
  
\begin{LTXexample}
\begin{tabular}{ll}
\includespread[range=b2:c]
\end{tabular}     
\end{LTXexample}  
\item[columns] Column heading specification. It can be either |head|, |top|, or comma separated list of values.
\begin{description}
\item[top] Use as headers first line from the table.
\begin{LTXexample}
\begin{tabular}{ll}
\includespread[range=b3:c5,columns=top]
\end{tabular}
\end{LTXexample}
Note that if you include whole table, first line is included twice:
\begin{LTXexample}
\begin{tabular}{lll}
\includespread[columns=top]
\end{tabular}     
\end{LTXexample}
in this case you can use
\item[head] use first row from selection as headings.
\begin{LTXexample}
\begin{tabular}{lll}
\includespread[columns=head,range=a:c3]
\end{tabular}     
\end{LTXexample}
\item[manually specified list] If column headings are not specified in the file, you can set them manually.
\begin{LTXexample}
\begin{tabular}{ll}
\includespread[columns=head,columns={title,amount},sheet=List2]
\end{tabular}     
\end{LTXexample}
\end{description}

\item[rowseparator] Rows are normally separated with newlines, if you really want, you can separate them with hlines
\begin{LTXexample}
\begin{tabular}{lll}
\includespread[columns=top,sheet=List1,rowseparator=hline,range=a2:b5]
\end{tabular}     
\end{LTXexample} 

\item[template] Templates are simple mechanism to insert whole tabular environment with column specification. All columns are aligned to the left, if you want to do more advanced stuff with column specifications, you must enter them manualy as in all previous examples.
\begin{LTXexample}
\includespread[columns=top,template=booktabs,range=a3]
\end{LTXexample}
For more info about templates, see next section \ref{sec:tpl}
\end{description}


\section{Templates}\label{sec:tpl}

If you don't want to specify tabular environment by hand, you can use simple templating mechanism to insert tabular environment by hand. 

Templates are defined with macro\marginpar{\cmd{\tabletemplate}}\\ 
\cmd{\tabletemplate}\marg{template name}\marg{template code}\\
there is default template:
\begin{verbatim}
\tabletemplate{default}{-{colheading}-{rowsep}-{content}}
\end{verbatim}

Code |-{variable name}| inserts one of the following variables:

\begin{description}
\item[coltypes] This is code to be inserted in |\begin{tabular}{coltypes}|. In current version, it inserts |l| for left alignment column, for all columns of inserted table. 
It should be possible to use more intelligent method based on types of column content, or \ods styles, maybe in future versions some of them will be used. If you want other alignment of columns now, you have to specify |\begin{tabular}{column types}| manually.
\item[colheading] Column headings.
\item[rowsep] It inserts row separator defined with |rowsepartor| key of |\includespread|. It is used in default style to delimit column headings and table contents.   
\item[content] Tabular data.
\end{description} 

\paragraph{More powerful template for the \textsc{booktabs} package}

\begin{verbatim}
\tabletemplate{booktabs}{%
\\begin{tabular}{-{coltypes}}
\\toprule
-{colheading}
\\midrule
-{content}
\\\\ \\bottomrule
\\end{tabular}
} 
\end{verbatim}
Note use of the double |\| in template definition -- it is needed to pass them to the |lua| side.
\section{Lua library}

The |lua| library uses |luazip| library integrated to \LuaTeX{} and |LuaXML|\footnote{\url{http://manoelcampos.com/files/LuaXML-0.0.0-lua5.1.tgz}}, pure |lua| library for working with |XML| files.

To use library in your code, you can use:

\begin{verbatim}
require("odsfile")
\end{verbatim} 

Function |odsfile.load(filename)| returns |odsfile| object, with |loadContent()| method, which returns |lua| table representing |content.xml| file. We can select sheet from the spreadsheet with |odsfile.getTable(xmlobject,sheet_name)|. If we omit |sheet_name|, first sheet from spreadsheet is selected.

Data from sheet can be read with |odsfile.tableValues(sheet, x1, y1, x2, y2)|. |x1 - y2| are range to be selected, they can be |nil|, in which case whole rows and cells are selected. For converting of standard range expressions of form |a1:b2| to this representation, function |odsfile.getRange(range)| can be used.

\paragraph{Example usage -- file \textsf{odsexample.lua}}

\begin{verbatim}
require "odsfile"

-- Helper function to print structure of the table
function printable(tb, level)
  level = level or 1
  local spaces = string.rep(' ', level*2)
  for k,v in pairs(tb) do
      if type(v) ~= "table" then
         print(spaces .. k..'='..v)
      else
         print(spaces .. k)
         level = level + 1
         printable(v, level)
      end
  end
end

local ods = odsfile.load("filename.ods")
local f, e = ods:loadContent()

-- Get First sheet from the table
body= odsfile.getTable(f)
-- Print structure of the range a4:b5  
printable(odsfile.tableValues(body,odsfile.getRange("a4:b5")))
\end{verbatim}

Run the example with |texlua odsexample.lua| from the command line, you should get following result:

\begin{verbatim}
  1
    1
      value=AA
      attr
        office:value-type=string
    2
        value=3
        attr
          office:value-type=float
          office:value=3
  2
      1
        value=BB
        attr
          office:value-type=string
      2
          value=4
          attr
            office:value-type=float
            office:value=4
\end{verbatim}

To convert this structure to \LaTeX{} tabular code, you can use following function:

\begin{verbatim}
function tableToTabular(values)
  local rowValues = function(row)
    local t={} 
    for _,column in pairs(row) do table.insert(t,column.value) end
    return t
  end
  content = {}   
  for i,row in pairs(values) do
    table.insert(content,table.concat(rowValues(row)," & "))
  end
  return table.concat(content,"\\\\\n")
end
-- Now use it with objects from previous example
print(tableToTabular(odsfile.tableValues(body)))
\end{verbatim}  

This example yelds

\begin{verbatim}
Label & Position & Count\\
Hello & 1 & 3\\
World & 2 & 4\\
AA & 3 & 5\\
BB & 4 & 6\\
CC & 5 & 7
\end{verbatim}
\end{document}